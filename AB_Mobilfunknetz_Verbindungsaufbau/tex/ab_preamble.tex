\usepackage[utf8]{inputenc}
\usepackage{lmodern}
\usepackage{graphicx}
\usepackage{setspace}
\usepackage{environ}
\usepackage{lipsum}
\usepackage{tabularx}
\usepackage[a4paper]{geometry} 
\geometry{top=25mm,left=25mm ,right=25mm, bottom=3cm} 
%
% =======================
% Abbildungstext anpassen
% =======================
\usepackage[normal,font={small}, labelfont=bf,figurename=Abb.]{caption}
%
% Pfad der Abbildungen
\graphicspath{ {../figures/} }
%
\setlength{\footheight}{50mm}
\ifoot{\footnotesize  Letzte Änderung: \today}
\ofoot{Seite Nr. \luecke{1cm}}
\ohead{Datum:\hspace*{3cm}}
%
% ----------
% ÜBERSHRIFT
% ----------
\newcommand{\Ueberschrift}[2]{
	\vskip 1.em
	\setlength{\tabcolsep}{0mm} % kein Innenrand bei Spaleten
	\begin{tabularx}{\linewidth}{lXr}
	{\Large\textbf{\textsf{#1}}} & &
	\includegraphics[height=1.5cm]{#2}\\ % Logo rechtsbündig
	\hline
	\end{tabularx}
	% Spaltenabstand zurücksetzen
	\setlength{\tabcolsep}{6pt} 
}
% ----------
% HINWEISBOX
% ----------
\NewEnviron{hinweisbox}[2]{%
\par
\vspace{1em}
\noindent
\begin{tikzpicture}[]
	\node[rectangle,minimum width=1.0\textwidth, inner sep=0pt] (m) {
		\begin{minipage}{\textwidth}
			\dimen0\linewidth
			\advance\dimen0 by -3.0em
    			\vskip .8em \hspace{1em}
    			\parbox{\the\dimen0}{
        			\BODY\vskip .5em}%			
		\end{minipage}
		};
	\draw[#2, very thick, rounded corners, color=#1] (m.south west) rectangle (m.north east);
\end{tikzpicture}
}
% ----------
% MERKE
% ----------
\NewEnviron{merke}{%
\par
\vspace{1em}
\noindent
\begin{tikzpicture}
	\node[rectangle,minimum width=1.0\textwidth, inner sep=0pt] (m) {
		\begin{minipage}{\textwidth}
			\dimen0\linewidth
			\advance\dimen0 by -3.0em
    			\vskip .8em \hspace{1em}
    			\parbox{\the\dimen0}{
        		\textbf{Merke:}	\BODY\vskip .5em}%			
		\end{minipage}
		};
	\draw[dotted, very thick, rounded corners, color=red] (m.south west) rectangle (m.north east);
\end{tikzpicture}
}
%
\newcommand{\Quelle}[1]{
	\ifoot{\footnotesize #1 \newline Letzte Änderung: \today }
}
%
% ----------
% Schwierigkeitsindikatoren
% ----------
\newcommand{\indikator}[1]{
	\begin{tikzpicture}[]
		\draw[green!50!black] circle (1.5mm);	
		\filldraw[fill=green!20!white, draw=green!50!black]
		(0,0) -- (1.5mm,0mm) arc (0: #1 :1.5mm) -- (0,0);	
	\end{tikzpicture}
}
%
\newcommand{\schwer}{
\begin{tikzpicture}[]
	\filldraw[fill=green!20!white, draw=green!50!black] (0,0) circle (1.5mm);
\end{tikzpicture}
}
\newcommand{\mittel}{
	\indikator{180}
}
\newcommand{\leicht}{
\begin{tikzpicture}[]
	\draw[green!50!black] (0,0) circle (1.5mm);
\end{tikzpicture}
}
% ----------
% AUFGABENBOX
% ----------
\NewEnviron{aufgabenbox}[2]{%
\par
\vspace{1em}
\noindent
\begin{tikzpicture}[]
	\node[rectangle,minimum width=1.0\textwidth, inner sep=0pt] (m) {
		\begin{minipage}{\textwidth}
			\dimen0\linewidth
			\advance\dimen0 by -3.0em
    			\vskip .8em \hspace{1em}
    			\parbox{\the\dimen0}{
        			\BODY\vskip .5em}%			
		\end{minipage}
		};
	\draw[solid, thin, color=black] (m.south west) rectangle (m.north east);
\end{tikzpicture}
}
% ----------
% StickyNote
% ----------
\usepackage{xparse}
\usepackage{fancypar}
\usetikzlibrary{shadows}
\definecolor{myyellow}{RGB}{242,226,149}
\makeatletter
\pgfdeclareshape{note}{
\inheritsavedanchors[from=rectangle] % this is nearly a rectangle
\inheritanchorborder[from=rectangle]
\inheritanchor[from=rectangle]{center}
\inheritanchor[from=rectangle]{north}
\inheritanchor[from=rectangle]{south}
\inheritanchor[from=rectangle]{west}
\inheritanchor[from=rectangle]{east}
% ... and possibly more
\backgroundpath{% this is new
% store lower right in xa/ya and upper right in xb/yb
\southwest \pgf@xa=\pgf@x \pgf@ya=\pgf@y
\northeast \pgf@xb=\pgf@x \pgf@yb=\pgf@y
% compute corner of ‘‘flipped page’’
\pgf@xc=\pgf@xb \advance\pgf@xc by-10pt % this should be a parameter
\pgf@yc=\pgf@yb \advance\pgf@yc by-10pt
% construct main path
\pgfpathmoveto{\pgfpoint{\pgf@xa}{\pgf@ya}}
\pgfpathlineto{\pgfpoint{\pgf@xa}{\pgf@yb}}
\pgfpathlineto{\pgfpoint{\pgf@xc}{\pgf@yb}}
\pgfpathlineto{\pgfpoint{\pgf@xb}{\pgf@yc}}
\pgfpathlineto{\pgfpoint{\pgf@xb}{\pgf@ya}}
\pgfpathclose
% add little corner
\pgfpathmoveto{\pgfpoint{\pgf@xc}{\pgf@yb}}
\pgfpathlineto{\pgfpoint{\pgf@xc}{\pgf@yc}}
\pgfpathlineto{\pgfpoint{\pgf@xb}{\pgf@yc}}
\pgfpathlineto{\pgfpoint{\pgf@xc}{\pgf@yc}}
}
}
\makeatother
%\setlist{leftmargin=*,itemsep=4pt,parsep=0pt}
%\renewcommand{\labelitemi}{-}
%
\NewDocumentCommand\StickyNote{O{6cm}mO{6cm}O{0}}{%
\begin{tikzpicture}
\node[
note,
draw,
drop shadow={
  shadow xshift=2pt,
  shadow yshift=-4pt
},
inner xsep=7pt,
fill=myyellow,
rotate=#4,
inner ysep=10pt
] {\parbox[t][#1][c]{#3}{#2}};
\end{tikzpicture}%
}

\NewDocumentCommand\StickyNotePi{O{6cm}mO{6cm}O{0}}{%
\begin{tikzpicture}
\node[
note,
draw,
fill=myyellow,
inner xsep=10pt,
rotate=#4,
inner ysep=0pt,
text depth=\the\dimexpr#1+2.5ex\relax
] {\parbox[t][#1][c]{#3}{#2}};
\end{tikzpicture}%
}
%
% -----------
% Schriftstil
% -----------
\renewcommand\familydefault{\sfdefault}