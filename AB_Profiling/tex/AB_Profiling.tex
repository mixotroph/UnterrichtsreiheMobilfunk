\documentclass[loesung]{schulein}
%
\kopfDatum{\today} 
\fach{NawI 9}
\dokName{Projekt: Profiler}
\keineSeitenzahlen
\usepackage[utf8]{inputenc}
\usepackage{lmodern}
\usepackage{graphicx}
\usepackage{setspace}
\usepackage{environ}
\usepackage{lipsum}
\usepackage{tabularx}
\usepackage[a4paper]{geometry} 
\geometry{top=25mm,left=25mm ,right=25mm, bottom=3cm} 
%
% =======================
% Abbildungstext anpassen
% =======================
\usepackage[normal,font={small}, labelfont=bf,figurename=Abb.]{caption}
%
% Pfad der Abbildungen
\graphicspath{ {../figures/} }
%
\setlength{\footheight}{50mm}
\ifoot{\footnotesize  Letzte Änderung: \today}
\ofoot{Seite Nr. \luecke{1cm}}
\ohead{Datum:\hspace*{3cm}}
%
% ----------
% ÜBERSHRIFT
% ----------
\newcommand{\Ueberschrift}[2]{
	\vskip 1.em
	\setlength{\tabcolsep}{0mm} % kein Innenrand bei Spaleten
	\begin{tabularx}{\linewidth}{lXr}
	{\Large\textbf{\textsf{#1}}} & &
	\includegraphics[height=1.5cm]{#2}\\ % Logo rechtsbündig
	\hline
	\end{tabularx}
	% Spaltenabstand zurücksetzen
	\setlength{\tabcolsep}{6pt} 
}
% ----------
% HINWEISBOX
% ----------
\NewEnviron{hinweisbox}[2]{%
\par
\vspace{1em}
\noindent
\begin{tikzpicture}[]
	\node[rectangle,minimum width=1.0\textwidth, inner sep=0pt] (m) {
		\begin{minipage}{\textwidth}
			\dimen0\linewidth
			\advance\dimen0 by -3.0em
    			\vskip .8em \hspace{1em}
    			\parbox{\the\dimen0}{
        			\BODY\vskip .5em}%			
		\end{minipage}
		};
	\draw[#2, very thick, rounded corners, color=#1] (m.south west) rectangle (m.north east);
\end{tikzpicture}
}
% ----------
% MERKE
% ----------
\NewEnviron{merke}{%
\par
\vspace{1em}
\noindent
\begin{tikzpicture}
	\node[rectangle,minimum width=1.0\textwidth, inner sep=0pt] (m) {
		\begin{minipage}{\textwidth}
			\dimen0\linewidth
			\advance\dimen0 by -3.0em
    			\vskip .8em \hspace{1em}
    			\parbox{\the\dimen0}{
        		\textbf{Merke:}	\BODY\vskip .5em}%			
		\end{minipage}
		};
	\draw[dotted, very thick, rounded corners, color=red] (m.south west) rectangle (m.north east);
\end{tikzpicture}
}
%
\newcommand{\Quelle}[1]{
	\ifoot{\footnotesize #1 \newline Letzte Änderung: \today }
}
%
% ----------
% Schwierigkeitsindikatoren
% ----------
\newcommand{\indikator}[1]{
	\begin{tikzpicture}[]
		\draw[green!50!black] circle (1.5mm);	
		\filldraw[fill=green!20!white, draw=green!50!black]
		(0,0) -- (1.5mm,0mm) arc (0: #1 :1.5mm) -- (0,0);	
	\end{tikzpicture}
}
%
\newcommand{\schwer}{
\begin{tikzpicture}[]
	\filldraw[fill=green!20!white, draw=green!50!black] (0,0) circle (1.5mm);
\end{tikzpicture}
}
\newcommand{\mittel}{
	\indikator{180}
}
\newcommand{\leicht}{
\begin{tikzpicture}[]
	\draw[green!50!black] (0,0) circle (1.5mm);
\end{tikzpicture}
}
% ----------
% AUFGABENBOX
% ----------
\NewEnviron{aufgabenbox}[2]{%
\par
\vspace{1em}
\noindent
\begin{tikzpicture}[]
	\node[rectangle,minimum width=1.0\textwidth, inner sep=0pt] (m) {
		\begin{minipage}{\textwidth}
			\dimen0\linewidth
			\advance\dimen0 by -3.0em
    			\vskip .8em \hspace{1em}
    			\parbox{\the\dimen0}{
        			\BODY\vskip .5em}%			
		\end{minipage}
		};
	\draw[solid, thin, color=black] (m.south west) rectangle (m.north east);
\end{tikzpicture}
}
% ----------
% StickyNote
% ----------
\input{StickyNote}
%
% -----------
% Schriftstil
% -----------
\renewcommand\familydefault{\sfdefault}
%
\begin{document} 
%\onehalfspace
%
\section*{Projekt: \textit{Profiler}}
Wir haben einen \textbf{Mobilfunkdatensatz} einer Person erhalten. Dieser erstreckt sich über den Zeitraum eines halben Jahres. Ziel des Projektes ist es ein Profil dieser Person zu erstellen, um so herauszufinden, um welche Person es sich handelt. Dafür gibt es sehr viele verschiedenen Ansätze und Lösungswege, eurer Kreativität sind dabei keine Grenzen gesetzt! Es ist jedoch zu beachten, dass sich das Stadtbild ständig ändert. Da der Datensatz aus dem Jahr 2009 ist kann es sein, dass sich verschiedenen Aufenthaltsorten bereits verändert haben. Dies solltet ihr bei eurer Recherche berücksichtigen!

%\vspace*{-1cm}
\Ueberschrift{Arbeitsauftrag}{ga}
%\par
%Der Kommunikationsaufbau im Mobilfunknetz 
\begin{aufgaben}
\item Überlegt zunächst welche Informationen ihr auf jeden Fall herausfinden möchtet. Strukturiert euren Profiling-Prozess so, dass ihr die Aufgaben unter euch verteilt! Dokumentiert euren Arbeitsprozess und die Aufgabenzuteilung!

\item Gestaltet ein Plakat, auf dem ihr eure Ergebnisse übersichtlich präsentiert. Lest dazu den Kasten \textit{Anforderungen an das Plakat}.  

\end{aufgaben}
%
\begin{aufgabenbox}{}{}
\vspace*{-1em}
\subsection*{Anleitung -- Starten von Processing und Öffnen der Datei
}
\begin{smallitemize}
\item Kopiert den ZIP-Ordner \texttt{viualisierung\_Handydaten.zip} und entpackt ihn in eurem Heimatverzeichnis.
\item Öffnet das Programm \textit{Processing} .
\item Klickt auf „File“ -> „open“.
\item Geht in eurer Heimatverzeichnis und öffnet aus dem entpackten ZIP-Ordner die Datei \texttt{visualisierung\_handydaten.pde}.
\item \textit{Processing} öffnet automatisch die Entwicklungsumgebung. In der Datei
findet ihr verschiedene Filter mit einer kurzen Anleitung.
\item Klickt oben links auf „Play“, um das Programm zu starten.
\item Im „Map-Bildschirm“ nochmals auf Play (unten mittig) drücken, dann wird das Programm ausgeführt und eure Filter angewendet.


\end{smallitemize}
\end{aufgabenbox}
%
\begin{aufgabenbox}{}{}
\vspace*{-1em}
\subsection*{Anforderungen an das Plakat}
Die strukturelle Zugehörigkeit von Informationen soll erkennbar sein. Das bedeutet: schreibt nicht einfach alles was ihr herausgefunden habt wahllos auf das Plakat. Teilt das Plakat ein. (Hierfür eignet sich eine kurze Brainstorming-Phase vor dem Start mit Processing und im mittleren Teil der Profiler-Tätigkeit, wenn ihr bereits einige Informationen herausgefunden habt).
\end{aufgabenbox}

\end{document}